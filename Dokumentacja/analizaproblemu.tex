\chapter{Analiza problemu}

\paragraph{}
Podobną funkcjonalność do opisywanego w dokumentacji systemu ma aplikacja webowa \textit{surveybee.net}. Oferuje ona nagrody pieniężne za wypełnienie ankiety, jednak minimalna stawka zarobionych pieniędzy, pozwalająca na wypłatę sprawia, że wypełniający musiałby poświęcić dużo czasu, aby otrzymać swoją nagrodę. Nasz system pozwala na natychmiastowe otrzymanie nagrody w postaci kuponu, bezpośrednio po wypełnieniu ankiety.

\paragraph{}
Innym systemem, znacznie bardziej popularnym i posiadającym większą reputację jest \textit{surveymonkey.com}. System ten pozwala firmom na tworzenie ankiet, które następnie mogą być wypełniane przez użytkowników. Aplikacja ta nie pozwala jednak na wprowadzenie gratyfikacji w zamian za wypełnienie ankiety.

\paragraph{}
Nasz system posiada zatem założenia funkcjonalne, którym różni się od produktów dostępnych na rynku. Implementacja takich założeń wiązała się z rozwiązaniem kilku problemów, które umożliwiałyby efektywne i bezpieczne wprowadzenie wspomnianych w rodziale pierwszym funkcjonalności.


\subsection{Problem liczby kuponów}
\paragraph{}
Jednym z najważniejszych założeń funkcjonalnych naszego systemu jest przekazywanie użytkownikom kuponów w zamian za wypełnienie ankiety. Problem ten jest trudny w sytuacji, gdy ilość kodów promocyjnych powiązanych z kuponami jest ograniczona. W przypadku naiwnego rozwiązania, polegającego na przydzielaniu kuponu wypełniającemu w momencie kliknięcia przycisku \texttt{Wyślij}, mogłoby dojść do sytuacji, w której ostatni kod zostałby przydzielony osobie, która szybciej rozwiązała ankietę. Osoba, która rozwiązywałaby ankietę wolniej po kliknięciu przycisku \texttt{Wyślij} zostałaby poinformowana o braku kodu promocyjnego. Jej czas przeznaczony na wypełnienie ankiety nie zostałby więc nagrodzony, a sam wypełniający poczułby się oszukany.

\paragraph{}
Aby uniknąć takiej sytuacji wprowadzone zostało blokowanie kuponów, które jest połączone z sesją użytkownika. Użytkownik po wybraniu interesującej go ankiety ma wiązany swój identyfikator sesji z kodem promocyjnym. W przypadku braku kodów promocyjnych dla danego kuponu, ankieta nie będzie widoczna dla użytkowników w systemie do momentu dodania większej ilości kodów promocyjnych. Dzięki takiemu rozwiązaniu użytkownik ma gwarancję otrzymania nagrody już na początku wybrania ankiety. Takie rozwiązanie wiąże się również z możliwościami paraliżowania systemu przez złośliwych użytkowników. W negatywnym przypadku złośliwy użytkownik mógłby zablokować wszystkie kody promocyjne dla siebie, przez co ankieta stałaby się niewidoczna dla innych użytkowników. Aby zapobiec takiej sytuacji wprowadzone zostały stosowne zabezpieczenia. Użytkownik systemu może w danym momencie mieć tylko jeden kod promocyjny powiązany z identyfikatorem sesji. Powiązanie te jest aktywne w systemie tylko przez 15 minut. Ponadto zamknięcie ankiety poprzez naciśnięcie przycisku \texttt{Anuluj} spowoduje zwrócenie kodu promocyjnego do puli. Dokładny opis procesu blokowania kuponu jest opisany w rodziale czwartym.

\subsection{Problem anonimowości}
\paragraph{}
W trosce o rzetelność odpowiedzi system zapewnia anonimowość odpowiadających. W bazie danych nie są przechowywane żadne dane osobowe, pozwalające na jednoznaczne zidentyfikowanie wypełniającego ankietę. Wszystkie dane osobowe, które mogą jednoznacznie wiązać użytkownika z konkretną osobą nie są przechowywane w bazie i są całkowicie opcjonalne. Przykładem takich danych jest adres e-mail, na który może zostać wysłany kupon po wypełnieniu ankiety. Jest on podawany opcjonalnie po wypełnieniu ankiety i nie jest nigdzie przechowywany. Innymi danymi zbieranymi od użytkownika są wiek oraz kraj pochodzenia, jednak informacje takie nie pozwalają wnioskować jednoznacznie o tożsamości wypełniającego.

\subsection{Problem bezpieczeństwa}
\paragraph{}
W systemie, który jako główne założenie funkcjonalne oferuje wartość materialną w zamian za wypełnienie ankiety, ważne jest zapewnienie bezpieczeństwa. Pierwszym problemem była walidacja odpowiedzi na ankiety. System należało zabezpieczyć przed atakami XSS zarówno przeprowadzanymi ze strony wypełniającego, jak i firmy. Aby ich uniknąć, wszystkie przesyłane pytania i odpowiedzi są walidowane przez serwer. Jako dodatkową walidację, zostało wprowadzone sprawdzanie tworzonych ankiet od strony front-endu. Nie jest to jednak zabezpieczenie wystarczające, aby uniemożliwić takie ataki, gdyż rozwiązania front-endowe mogą być modyfikowane od strony klienta. System jest również zabezpieczony przed atakami typu CSRF poprzez użycie tokenów CSRF. Szczegóły tego rozwiązania zostały opisane w rodziale trzecim. Zabezpieczone zostały również komponenty systemu, do których dostęp powinny mieć tylko zalogowane firmy, dzięki zastosowaniu podziału na role i autoryzacji w systemie. To rozwiązanie również zostało opisane dokładnie w rodziale trzecim. Dane firmy przechowywane są w bazie danych w sposób minimalizujący straty w razie ewentualnego wycieku danych. Hasła firm trzymane są w postaci hasha z solą, dzięki czemu adwersarz porównujący hasła nie jest w stanie stwierdzić identyczności między hasłami poszczególnych firm. Również proces odzyskiwania hasła został zaprojektowany w sposób bezpieczny, który uniemożliwia przejęcie kontroli nad kontem firmowym osobie trzeciej.